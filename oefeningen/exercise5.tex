\documentclass{../qh_exercise}
\graphicspath{ {./images/} }

\begin{document}

\section{Leerdoelen}
\begin{itemize}
    \item Het maken en gebruiken van classes
    \item Het gebruiken van objecten (instances of classes)
    \item Het maken van methods
    \item Het werken met een ArrayList
\end{itemize}

\section{Uitleg}
\begin{itemize}
    \item \myhref{https://www.youtube.com/watch?v=YcbcfkLzgvs&list=PLRqwX-V7Uu6bb7z2IJaTlzwzIg_5yvL4i}{8: Object-Oriented Programming}
    \item \myhref{https://www.youtube.com/watch?v=NptnmWvkbTw&list=PLRqwX-V7Uu6bO9RKxHObluh-aPgrrvb4a&index=1}{9: Arrays en ArrayList}
    \item \myhref{https://natureofcode.com/book/chapter-4-particle-systems/}{Particle Systems - The Nature of Code}\\
    Alleen \textbf{4.1, 4.2, 4.3, 4.4, 4.5}
    \item \myhref{https://www.youtube.com/watch?v=vdgiqMkFygc&list=PLRqwX-V7Uu6Z9hI4mSgx2FlE5w8zvjmEy&index=1}{Chapter 2. Particle Systems}\\
    Alleen \textbf{4.1, 4.2, 4.3, 4.4, 4.5}\\
\end{itemize}

\section{Voorbeelden}
\begin{itemize}
    \item clouds
    \item ball
\end{itemize}

\newpage
\section{Opdrachten}
Deze week gaan we voor het eerst \textit{beweging} maken! Voortaan is het belangrijk dat je alleen maar tekent in de \texttt{draw} functie. Het is extra belangrijk dat je de uitlegvideo's over Object G\"eorienteerd Programmeren!

\subsection{Een class}
Het is super handig om bepaalde variabelen en functies op die variabelen samen in \'e\'en object te bundelen. Hiervoor gebruiken we een \textbf{class}. In een class kun je meerdere variabelen en functies stoppen. Zorg ervoor dat je goed snapt hoe de volgende sketch werkt:
\begin{lstlisting}
Ball ball1;
Ball ball2;

void setup() {
    size(500,500);
    ball1 = new Ball(new PVector(width / 2, height / 2));
    ball2 = new Ball(new PVector(width / 2, 0));
}

void draw() {
    background(255);
    ball1.move();
    ball1.draw();
    ball2.move();
    ball2.draw();
}

class Ball {
    // Variabelen van deze class
    PVector pos = new PVector(width / 2,height / 2);
    PVector gravity = new PVector(0,5);
    
    // De constructor
    Ball(PVector beginPos) {
        this.pos = beginPos;
    }
    
    void move() {
        pos.add(gravity);
    }
    
    void draw() {
        circle(pos.x,pos.y,50);
    }
}
\end{lstlisting}
Pas de code aan zodat je naast de begin positie van de bal, ook de kleur kan aangeven.

\newpage
\subsection{Stuiterballen}
Pas de sketch aan door het volgende toe te voegen (en de method \texttt{move} te vervangen).
\begin{lstlisting}
    PVector bounceForce = new PVector(0,0);

    void move () {
        bounceIfBottom();
        pos.add(grivity);
        pos.add(bounceForce);
        bounceForce.mult(0.9);
    }

    void bounceIfBottom() {
        //TODO
    }
\end{lstlisting}
Deze method moet de bal laten stuiteren als deze de onderkant van het scherm raakt. 

\subsection{Een ArrayList}
In het de sketch van de vorige opdracht worden er twee variabelen gebruikt (\texttt{ball1} en \texttt{ball2}) om de ballen op te slaan. Maar wat nu als we 4 ballen op willen slaan, of 100000? We willen niet een hele lijst met variabelen maken natuurlijk. Daarom gebruiken we een \texttt{ArrayList}. Pas je gemaakte sketch aan zodat je met behulp van een \texttt{ArrayList} zoveel bellen kan maken als je wil!
Pas vervolgens de code aan zodat je 100 ballen op willekeurige plekken maakt!
\tip{Je krijgt een willekeurig getal door \texttt{int(random(maximum\_hoeveelheid))}}



\end{document}