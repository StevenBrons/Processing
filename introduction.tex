\documentclass[./syllabus.tex]{subfiles}

\begin{document}

\section{Inleiding}
Wat leuk dat je meedoet aan de \textbf{Q-Highschool} module \textbf{Processing}. In deze module zul je gaan werken in het Processing framework geschreven in \textbf{Java}. In deze module focussen we visueel programmeren, ofwel, code schrijven die iets op een scherm laat zien. Met deze modulen proberen we de \textit{magie van code} te laten zien, dat code niet alleen tekst is, maar dat het gebruikt kan worden om mooie en leuke dingen te maken. We wensen alle deelnemers van deze module veel plezier!

\section{Opbouw module}
Deze module bestaat uit 8 lesweken. In de eerste 6 weken leer je nieuwe concepten die je in de laatste twee weken gaat toepassen in een grotere opdracht. Er wordt veel gebruik gemaakt de video's en de website van \textbf{Daniel Shiffman}. Het is aan jezelf de bronnen die aangeleverd worden ook te gebruiken. Ook is per week een lijst met voorbeeld-programma's. We raden zeer sterk aan dat je deze bekijkt, een beetje aanpast en kijkt of je snapt wat er gebeurd. Verder zijn er veel oefenopdrachten. Om de stof goed te begrijpen raden we sterk aan deze opdrachten serieus te maken. Alle (voorbeeld) oplossingen van deze opdrachten zijn gegeven. Probeer het altijd eerst zelf voordat je naar de oplossingen kijkt. Om de twee weken is er een \textbf{inleveropdracht} (dus 4 in totaal). Deze worden door voor een cijfer beoordeeld.

\section{Kennismaking}
\textbf{TODO}

\section{Inspiratie}
\textbf{TODO}

\end{document}