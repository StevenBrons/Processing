\documentclass{article}
\usepackage[utf8]{inputenc}
\usepackage{hyperref}

\usepackage{listings}
\usepackage{color}
\usepackage{amsmath}
\usepackage{wrapfig}
\usepackage{graphicx}
\graphicspath{ {./images/} }
\usepackage{tikz}

\definecolor{mygreen}{rgb}{0,0.6,0}
\definecolor{mygray}{rgb}{0.5,0.5,0.5}
\definecolor{mymauve}{rgb}{0.58,0,0.82}

\lstset{ %
  backgroundcolor=\color{white},   % choose the background color
  basicstyle=\footnotesize,        % size of fonts used for the code
  breaklines=true,                 % automatic line breaking only at whitespace
  captionpos=b,                    % sets the caption-position to bottom
  commentstyle=\color{mygreen},    % comment style
  escapeinside={\%*}{*)},          % if you want to add LaTeX within your code
  keywordstyle=\color{blue},       % keyword style
  stringstyle=\color{mymauve},     % string literal style
  numbers=left,               % Ort der Zeilennummern
  language=java,
}

\newcommand{\myhref}[2]{\href{#1}{#2}\\\texttt{#1}}

\begin{document}

\title{\tttt}
\author{Steven Bronsveld}

\maketitle
\section{Gegevens}

\subsection{Algemene bronnen}
\begin{itemize}
    \item \myhref{http://www.github.com/StevenBrons/Processing}{GitHub Pagina} 
    \item \myhref{https://natureofcode.com/book}{Nature of Code boek} 
    \item \myhref{https://www.youtube.com/user/shiffman/playlists?view=50\&sort=dd\&shelf\_id=2}{Processing Basic tutorial}
\end{itemize}


\newpage
