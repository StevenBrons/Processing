\newcommand{\tttt}{Recursie}
\newcommand{\dddd}{Datum 1}

\documentclass{article}
\usepackage[utf8]{inputenc}

\usepackage{listings}
\usepackage{color}
\usepackage{amsmath}
\usepackage{wrapfig}
\usepackage{graphicx}
\graphicspath{ {./images/} }
\usepackage{tikz}

\definecolor{mygreen}{rgb}{0,0.6,0}
\definecolor{mygray}{rgb}{0.5,0.5,0.5}
\definecolor{mymauve}{rgb}{0.58,0,0.82}

\lstset{ %
  backgroundcolor=\color{white},   % choose the background color
  basicstyle=\footnotesize,        % size of fonts used for the code
  breaklines=true,                 % automatic line breaking only at whitespace
  captionpos=b,                    % sets the caption-position to bottom
  commentstyle=\color{mygreen},    % comment style
  escapeinside={\%*}{*)},          % if you want to add LaTeX within your code
  keywordstyle=\color{blue},       % keyword style
  stringstyle=\color{mymauve},     % string literal style
  numbers=left,               % Ort der Zeilennummern
  language=java,
}

\begin{document}

\title{\tttt}
\author{Steven Bronsveld}

\maketitle
\section{Gegevens}
Uiterlijke inleverdatum: \textbf{\dddd}
\subsection{Links}
\begin{itemize}
    \item Github.com/StevenBrons 
    \item https://natureofcode.com/book
    \item http://hello.processing.org/editor/
\end{itemize}


\newpage


\section{Leerdoelen}
\begin{itemize}
    \item Methods met \texttt{return} van type \texttt{int}
    \item Printen naar het console door middel van \texttt{System.out.println()}
    \item Het recursief aanroepen van methods
    \item Het recursief tekenen van simpele fractels
\end{itemize}
\section{Recursie}
\subsection{Recursieve functies}
Recursieve functies zijn functies die een (makkelijkere) versie van zichzelf gebruiken voor het bereken van een antwoord. Kijk bijvoorbeeld naar:
\[f(g,n) = \begin{cases}
        $1 \qquad\qquad\qquad n = 0$\\
        $g * f(n-1)  \qquad n $>$ 0$
    \end{cases}
   \]
Deze functie geeft $f(g,n) = g^n$. Als we dit uitschrijven krijgen we: $f(3,4) = 3 * f(3) = 3 * 3 * f(2) = 3 * 3 * 3 * f(1) = 3 * 3 * 3 * 3 * f(0) = 3 * 3 * 3 * 3 * 1 = 81 = 3^4$
\[g(n) = \begin{cases}
        $1 \qquad\qquad\qquad n = 0$\\
        $n * g(n-1)  \qquad n $>$ 0$
    \end{cases}
   \]
Schrijf de uitwerking van $g(5)$ helemaal uit. Weet je ook welke functie $g$ is?
\[h(n) = \begin{cases}
        $1 \qquad\qquad\qquad n = 0$\\
        $1 \qquad\qquad\qquad n = 1$\\
        $h(n - 1) + h(n - 2)  \qquad n $>$ 1$
    \end{cases}
   \]
Schrijf de uitwerking van $h(4)$ helemaal uit. Weet je ook welke functie $h$ is?

\subsection{Factorial}
Maak een recursieve functie die de \textit{factorial} van een getal berekent. (Zie \texttt{f} van de vorige opdracht). 
\begin{lstlisting}
int factorial(int n) {
}
\end{lstlisting}
Roep de functie aan in \texttt{setup} en print het resultaat. 

\subsection{Codeer Koch's Curve}
Maak een functie 
\begin{lstlisting}
    void KochCurve(int n,int x, int y, int l) {
    }
\end{lstlisting}
die Koch's Curve \texttt{n} diep, beginnend op \texttt{x,y} en eindiged op \texttt{x+l,y}.

\subsection{Codeer een Binary Tree}
Maak een method
\begin{lstlisting}
void binaryTree(int n,int x, int y) {
}
\end{lstlisting}
Die een \textbf{binary tree} tekent van \texttt{n} diep. Op de co\"ordinaten \texttt{x, y}.

\subsection{Codeer een Sierpinski Triangle}

\subsection{[Bonus] Koch's Snowflake}

\subsection{[Bonus] Hilbert's Curve}


\section{Inleveren}
Als je klaar bent met de hele opdracht kun je deze naar je \textit{repository} \texttt{pushen}.



\end{document}
